% Copyright © 2012 Edward O'Callaghan. All Rights Reserved.

\subsection{X Server} % (fold)
\label{sec:xserver}

The \emph{X Server}, much like the terminal emulator, provides graphical
windowing functionality as the fundamental object of control. This is in
contrast with the terminal emulator that instead uses shells that use
individual commands as the fundamental objects of control. Notice the
distinction! The X server windowing system \emph{environment} is where
graphical client applications can ask the X server to draw them a window
from which you can control the console though the mouse. Conversely, the
shell asks the terminal emulator to output results from commands and takes
input from the console though the terminal emulator.

In summary:
\begin{itemize}
	\item A terminal emulator \emph{abstracts} the console away from commands
		in a shell.
	\item The X server \emph{abstracts} the drawing of graphical objects on the
		console away from the client applications.
\end{itemize}

\subsubsection{Toolkits}

Applications typically want to draw more complex things such as triangles and
squares. However the X server does not attempt to draw sophisticated graphical
objects, keeping true to the GNU/Linux philosophy. In this way the X server provides
the flexibility to draw whatever kind of object you would like in a graphical way
on the screen. However, applications typically would like to draw the same kind of
thing over and over and that various applications on a graphical desktop would
ideally like to look consistent from a usability prospective. Hence, on a typical
GNU/Linux installation one normally installs a \emph{toolkit} to go with the X server
and the preferred range of applications that make use of the said toolkit. Toolkits
provide another abstraction on top of the X server to draw graphical applications to
look in a consistent and feature full way.

\begin{exmp}[Some common X toolkits]
	Some example toolkits that are typically found;
	\begin{itemize}
		\item Gtk2/3,
		\item Qt,
		\item Clutter.
	\end{itemize}
\end{exmp}

The X server is not the only place graphical toolkits are found. In fact, in the shell
one can draw \emph{ASCII} art type graphical interfaces though a common toolkit known
as \textbf{ncurses}. Ncurses once again abstracts away the terminal emulators ability
to draw ASCII text to the screen and provides a rang of \emph{API}'s, or
Application Programmer Interface's, that allow programmers to draw graphical interfaces
in the form of text on the console though a terminal emulator.

\subsubsection{Window Manager}

The window manager is essentially the \emph{job control} of the graphically world on
a GNU/Linux system. The window manager is the main program the X server runs that
controls how applications behave with one another and how they are presented to the
user in terms of method of control. Typically the window manager is what the user
perceives to be \emph{the desktop}! So the window manager is a important choice
as to how you prefer to work and prefer your desktop to act and look. The window
managers environment is analogous to the shell language that comes with the shell
of choice. Hence, desktop environment settings such as setting your wallpaper is
equivalent to setting, say for example, the \emph{PATH} environment variable of
your shell.

\begin{exmp}[Some common window managers]
	Some example window managers that are popular;
	\begin{itemize}
		\item XMonad,
		\item KDE,
		\item Gnome.
	\end{itemize}
\end{exmp}

Typically two main paradigms exist for window managers, \textbf{tilling} and \emph{stacking}.
The most popular of which is typically the stacking paradigm in which program windows are
layer on top of each other. Both Gnome and KDE are example window managers that are of the
stacking sort, although can be manually configured otherwise. However the very minimalistic
XMonad window manager is extremely small, uses almost no memory at all (i.e., less than 1MB)
and is of the tiling sort. The main advantage of tiling window managers comes in when you are
running more than a few programs and spend more time switching between programs than actually
doing your intended work inside the programs. Another advantage of the usually minimal nature
of tiling window managers is the desktop is not cluttered with random applications popping out
everywhere and arbitrarily large unnecessary boarders around menu parts taking up valuable
screen real estate.

\subsubsection{Login Manager}

To start the X server one can run \emph{startx} as the intended user and configure which window
manager the X server should start in \emph{.xinitrc} in their home directory. However, on a
desktop workstation these days you typically wish for the X server to start immediately and be
presented a graphical method of login. This is the intended task of a login manager, at boot
the system start the graphical login manager as a system service and the login manager starts
the X server in turn. Various login managers exist.

\begin{exmp}[Some common login managers]
	Some example login managers, some of which come with the window manager for a more
	transparent experience.
	\begin{itemize}
		\item KDM - comes with the KDE window manager,
		\item GDM - comes with the Gnome window manager,
		\item SLiM - a light weight login manager that is independent of any particular window manger.
	\end{itemize}
\end{exmp}

A good match with the XMonad window manager is the SLiM login manager. Since SLiM is very
minimal and does not reply on the toolkit libraries Qt or Gtk3 from KDE or GDM respectively.
