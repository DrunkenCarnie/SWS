% Copyright © 2012 Edward O'Callaghan. All Rights Reserved.

\section{Networking} % (fold)
\label{sec:networking}

\subsection{Basics}

Basic networking starts with setting up what the machines
\emph{hostname} should be. This is some arbitrary name we
give to the physical machine console on the network. To
specify the name of the machine, simply make something up
and do the follow:
\begin{center}
	\$ echo "starkid" $>$ /etc/hostname
\end{center}
You will also need to append this name in /etc/hosts as
an alias to \emph{localhost}. The \emph{localhost} is the
default name that simply means the local machine.

A \emph{fully qualified} \emph{username} is then of the form:
\begin{center}
	username@hostname
\end{center}
For example, joebobs@starkid.

\subsection{Dynamic IP}

\subsection{Static IP}

\subsection{Wired}

\subsection{Wireless}
