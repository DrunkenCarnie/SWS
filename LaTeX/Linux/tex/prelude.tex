% Copyright © 2012 Edward O'Callaghan. All Rights Reserved.

\section{Prelude} % (fold)
\label{sec:prelude}

Typically a desktop computer consists of three main components;
\begin{enumerate}
	\item The physical hardware.
	\item The supervisor operating system or \emph{kernel}.
	\item The client applications, (i.e., web browsers, text editors,..).
\end{enumerate}

There are various choices in operating system supervisors around, such as,
to name a few;
\begin{itemize}
	\item Microsoft Windows NT (found in Windows).
	\item Darwin (found in Apple OSX).
	\item BSD.
	\item Linux (typically GNU/Linux).
\end{itemize}

Typically we do not interact with the supervisor, or kernel, directly but
though client applications. Hence, the operating system is typically
misconstrued as being the installed client applications when this is
in fact not the case. In this course we shall study the GNU/Linux
operating system and consider some usual client applications and their
configuration.

Various GNU/Linux \emph{distributions} exist, each of which tailored
to a particular audience. The \textbf{ArchLinux} distribution is
favoured for its neutrality of not imposing any particular pre-configuration
on the user. In ArchLinux you are required to configure everything yourself
how you wish. In this way, ArchLinux is somewhat true to the spirit of
GNU/Linux's simplicity and individuality is better in regards to program
design. The ArchLinux distribution simply provides a few various scripts
to get it installed and a core bootable system and package manager to allow
you to install your preferred software without imposing any choice on you.

\begin{exmp}[Various common GNU/Linux distributions]
	ArchLinux is not the only distribution of course. Some common
	alternatives are listed;
	\begin{itemize}
		\item Red Hat Fedora Linux,
		\item Novel SuSE Linux,
		\item Slackware Linux,
		\item Debian Linux,
		\item Ubuntu Linux (made from Debian).
	\end{itemize}
\end{exmp}
