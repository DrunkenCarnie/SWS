% Copyright © 2012 Edward O'Callaghan. All Rights Reserved.

\section{Installation} % (fold)
\label{sec:installation}

\subsection{Disk Partitions}

Disk partitions are essentially the way in which we may divide
up the disk. For example, we may wish for our personal files
in our \emph{home} directory to be separate from the rest of the
system files. By partitioning the disk we may \emph{format} each
partition separately, typically with different filesystems. The
advantage in the case of the \emph{home} directory having a different
filesystem is that of performance and increased safety. For example,
if, for whatever reason, our system was unbootable and irreparable
(broken) our files are safe from doing a clean install without the
worry of overwriting them. Although it is \textbf{always} a good
idea to regularly backup important personal data and configuration
files.

\subsection{File-systems}

The filesystem is the software to which the Linux kernel
uses when reading and writing data to some storage media
such as a disk. The filesystem is essentially a algorithmic
software description as to how to organise data on the storage
media in an optimal and consistent way. Hence, we typically say
that a disk has been \emph{formated} with a particular filesystem.
For example, ``the USB pen has been formated with FAT32''.

\begin{exmp}[Common Filesystems]
	Here is a short list of commonly found Linux filesystems:
	\begin{itemize}
		\item XFS,
		\item JFS,
		\item EXTended 2,3 and 4,
		\item BTFS,
		\item ZFS,
		\item procfs,
		\item tempfs,
		\item swapfs.
	\end{itemize}
	Some common filesystems found both in Linux and other places
	such as Microsoft Windows and Apple OSX are given:
	\begin{itemize}
		\item FAT32,
		\item NTFS,
		\item HFS+.
	\end{itemize}
\end{exmp}

The EXTended filesystem, or \emph{ext}, is the usual default Linux
filesystem software. However, ethier JFS or XFS is recommended for
its greater maturity and hence stability and reliability. Something
of a critical concern considering the filesystem takes care of your
data! USB storage devices are usually formated with FAT32 since
almost every system can read and write this very basic filesystem
formatting and hence is very compatible, however it is very minimal.

In Windows, NTFS is the default filesystem and you have little to no
choice about that. NTFS is proprietary software and so if something goes
wrong you really have no possible way to find out how NTFS went wrong
since we have no access to its code. Apple's OSX uses HFS+ which has its
own set of problems and is becoming somewhat long in the tooth. The most
critical point about proprietary filesystem software, like NTFS, is that
your data is trapped in this format and only systems that have access to
the code that makes NTFS work may read and write to these filesystems. In
this way, proprietary software traps you and your rights to your data!
Worse still is that, if you recall, propriary software licensing is typically
written so that you buy the rights to \emph{use} the software not \emph{own} it.
Hence, by storing your data on filesystems like NTFS you only have the rights
to \emph{use} your data even though you may own your data you may not always
have the right to \emph{access} it!
