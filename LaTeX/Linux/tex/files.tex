% Copyright © 2012 Edward O'Callaghan. All Rights Reserved.

\section{Files} % (fold)
\label{sec:files}

First consider the following proposition and remember this as it
is essentially the whole design premise of the Linux operating system!

\begin{prop}
	Everything in Linux is consider a file of two main kinds:
	\begin{itemize}
		\item Block, or
		\item Character Stream.
	\end{itemize}
	A \textbf{Block} is a contiguous chunk of binary data whereas
	a \textbf{Stream} is a flow of binary data bit by bit where a
	bit is one or zero.
\end{prop}

Now we may consider some concrete examples in order to fix our ideas
of what this actually means in practice.

\begin{exmp}[Block Device]
	A block device or block file comes up in various places such as;
	\begin{enumerate}
		\item A hard disk.
		\item A USB pen (data is buffered in VFS since the USB port is serial).
		\item The screen ``framebuffer''.
		\item A file on the disk.
		\item A allocated chunk of memory can be consider a block of memory.
	\end{enumerate}
\end{exmp}

\begin{exmp}[Character Device]
	A character device is a very typical type of file used to stream
	binary content to some kind of media such as;
	\begin{enumerate}
		\item A printer.
		\item A serial port, (i.e., RS232, i2c, USB).
		\item Audio jack out.
		\item A Linux pipe stream.
	\end{enumerate}
\end{exmp}

The Linux kernel deals with files by a layer called \textbf{VFS} or
\emph{Virtual File System}. The detail of this is not important to
understand how things work in general.

Three main special stream device files are used to interact with the system:
\begin{defn}[Standard Input/Output (IO)]
	The \emph{Standard IO} streams are how programs talk with the screen
	and other programs, in this case programs interact in a general way
	with the following standard IO streams:
	\begin{enumerate}[1] % FIXME: Start at zero?
		\item Standard Input or /dev/stdin
		\item Standard Out or /dev/stdout
		\item Standard Error or /dev/stderr
	\end{enumerate}
\end{defn}

Consider some web browser application that would like to print a page
then save it. The browser need not known anything about the type of
printer or type of media it is storing files to. Only that it needs to
\emph{write} something to a printer and \emph{write} something to a file.
This is exactly how applications interact with VFS in the Linux kernel.

A application simply writes to a printer file or writes to a disk file
and VFS \textbf{abstracts} the writing operation to whatever subsystem
takes care of the detail of the write.

In Linux printer files are represented though a userland program called
CUPS or \emph{Common Unix Printer System} which presents \emph{printer files}
that are software block devices, called a \emph{spool}, that buffer the
page and streams it to the physical device.

In the situation where we are writing a file to a disk, VFS now presents
a software block device that \emph{represents} the physical disk. In
particular, VFS takes the content of the software block device buffer,
finds the correct filesystem software that the disk is formated with
and uses this to correctly write the data to the physical media in the
respective filesystem formating. Some example disk filesystems are;
XFS, JFS, NTFS, FAT32, etc..

\begin{rem}
	Notice that VFS represents physical storage media as software
	buffers in memory. This is why we must \emph{umount} a disk
	such as a USB storage pen \textbf{before} physically removing
	it from the system. Since, when we write to the software
	representation we are \emph{not} writing to the actual disk
	and so we \emph{must} ensure that everything in the software
	buffer has been flushed or \emph{sync}'ed to the physical disk!
	Failure to do this may result in loss of data!
\end{rem}

Similarly to the printer situation, in the case of audio mixing
Linux presents the sound card by an equivalent software representation.
This particular representation is achieved by \emph{ALSA} or the
Advanced Linux Sound Architecture. That is, when playing a song we
write the song file to ALSA's representation of the sound card
(i.e., a software buffer block device) and ALSA takes care of reading
this block of data out of memory bit by bit and streaming that to the
physical sound card.

A somewhat special file exists that you may of considered by now, a
\emph{directory}. A directory is simply a file that contains the location
of all the files it has assigned as children. This design choice naturally
leads to the representation of all files on the system as a tree-like structure.
Note that the structure is not a pure tree, since some files are software links
to other files or essentially a directory with one and only one entry. Hence,
the structure in actual fact is more like a graph on a real system. In any
case this should provide us with the understanding of why \emph{mount} points
are required.

A \emph{mount point} is simply a directory in which we mount the software file
representation from VFS of a physical disk. The directory we mount to simply
has the pointer to location of the memory where VFS looks for data to read and
write to a physical disk. When we call the command \textbf{mount} we are writing
this address into the directory used as the mount point. When we call the command
\textbf{umount} we are forcing the data in VFS's buffer to be flush to the physical
disk and then removing the address entry from the original mount point directory.

\subsection{File-system Heredity}
..

\subsection{File Manipulation}

\begin{defn}[Moving files]
	To move a file we use the command \textbf{mv}.
\end{defn}

\begin{defn}[Copying files]
	To copy a file we use the command \textbf{cp}.
\end{defn}

\begin{defn}[Renaming files]
	To rename a file we use the command \textbf{rename}.
\end{defn}

\begin{defn}[Removing files]
	To remove a file we use the command \textbf{rm}.
\end{defn}
